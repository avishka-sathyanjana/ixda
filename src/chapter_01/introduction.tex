\section{Introduction}
\par Affective computing is a multidisciplinary research area where computer science bridges the gap between cognitive science, psychology, and social science. It empowers intelligent systems to recognize, conclude, and interpret human emotions, facilitating better human-machine interaction by responding to
humans based on their emotional state \citep{picard2000affective}. Given the significant variability of emotional states across individuals, personalized emotion recognition models are essential, particularly for applications such as personal assistants where user satisfaction and engagement are paramount \citep{salama2020deep}.

\par However, despite advances in real-time emotion recognition, current systems often provide generic responses that fail to account for the dynamic and personalized nature of emotional experiences \citep{kim2024understanding}. This limitation arises because emotions are rooted in an individual's baseline emotional state, which represents their stable mood \citep{davidson1998affective}, yet most systems overlook this baseline, resulting in responses that may feel disconnected or inadequate. The root of this issue lies in the lack of longitudinal analysis and personalized baselines in existing systems, which compromises the accuracy of emotion identification by failing to account for individual emotional biases. 

\par To address this gap, this study develops a personalized multimodal emotion recognition framework that integrates facial and vocal signals to establish and refine individual emotional baselines. By employing Decision Level Fusion and reinforcement learning, the framework achieves greater accuracy in capturing emotions while adapting to individual differences. This approach not only enhances the precision of emotion recognition but also fosters responsive, empathetic, and adaptive interactions, with significant implications for domains such as mental health support, customer service, and adaptive learning. The research objectives are to develop this framework, evaluate its effectiveness in recognizing personalized emotional states, and assess its impact on the quality of responses generated by large language models.