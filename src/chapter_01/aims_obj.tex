\section{Research Aim, Questions and Objective}
\label{sec:research-aim-questions-objectives}

\subsection{Research Aim}
\label{subsec:research-aim}
\par To develop and evaluate a personalized emotion recognition framework that uses facial expressions and audio data to identify an individual's emotional baseline, with the goal of enhancing the emotional intelligence of LLM responses.

\subsection{Research Questions and Objectives}
\label{subsec:research-questions-objectives}

\textbf{Objective 1 - Develop a multi-modal emotional recognition framework with Personalized Arousal-Valence Identification}
\label{obj:1}    
\begin{itemize}
    \item \textbf{RQ 1.1:}\label{rq:1.1} What are suitable pre-implemented models that can be used to get a higher accuracy for emotion recognition?

    \item \textbf{RQ 1.2:}\label{rq:1.2} How the recognized emotion values from different modalities fused together in order to get more personalized arousal-valence value?
      
    \item \textbf{Approach:} Conduct a comprehensive literature review to identify suitable pre-trained ML models along with the suitable datasets for each modality. Participants will engage in emotion-eliciting tasks to gather real-time data, enabling fine-tuning of the fusion technique. User feedback will guide adjustments, refining the model to accurately capture personalized arousal-valence values, thereby enhancing the system's adaptability to individual emotional responses.
\end{itemize}

\textbf{Objective 2 - Identify Initial Baseline and Implement Dynamic Personalized Baseline Identification Using Reinforcement Learning Iterations}
\label{obj:2}
\begin{itemize}
    \item \textbf{RQ 2.1:}\label{rq:2.1} What techniques are most suitable for establishing an initial emotional baseline and how can this baseline be dynamically adjusted over time to reflect changes in the user's emotional responses and self-reported feedback?
    
    \item \textbf{Approach:} Begin by evaluating various techniques to determine the best fit for identifying an initial emotional baseline using data from emotional-eliciting tasks. Implement an iterative adjustment process where participant feedback and ongoing data inputs help refine the baseline over time. This approach enables the model to dynamically adapt to individual users.
\end{itemize}


\textbf{Objective 3 - Integrate the personalized emotional state information with user queries and measure the impact on the quality of responses generated by a LLM, compared to a control condition without emotional state input}
\label{obj:3}
\begin{itemize}
    \item \textbf{RQ 3.1:}\label{rq:3.1} How does integrating personalized emotional state information with user queries affect the relevance and user satisfaction of responses from LLMs?
    
    \item \textbf{Approach:} Participants will interact with a LLM by providing queries and receiving two responses: one from passing the raw query to the LLM (control condition), and another one combining the personalized emotional state information along with the user query (experimental condition).
\end{itemize}